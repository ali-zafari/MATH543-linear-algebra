\documentclass[12pt, letterpaper]{scrartcl}


\usepackage{fullpage} % Set margins and place page numbers at bottom center
\usepackage[shortlabels]{enumitem} % Use a. in the enumerate
\usepackage{amsmath} % aligned equations
\usepackage{graphicx} % include figure
\usepackage{float} % usage of H for figure float
\usepackage{amssymb} % \blacksqure
\usepackage{xhfill} % fill horizontal line

\usepackage{colortbl}
\usepackage{xcolor} % colors
\usepackage{sectsty} % section coloring
\sectionfont{\color{blue}}  % sets colour of sections

%%%%%%%%%%%%%%%%%%%%%%%%%%%%%%%%%%%%%%%%%%%%%%%%%%%%%%%%%%%%%%%%%%%%%%
% MY COMMANDS                                                        %
%%%%%%%%%%%%%%%%%%%%%%%%%%%%%%%%%%%%%%%%%%%%%%%%%%%%%%%%%%%%%%%%%%%%%%
\newcommand{\Z}{\mathbb{Z}}
\newcommand{\R}{\mathbb{R}}
\newcommand{\C}{\mathbb{C}}
\newcommand{\F}{\mathbb{F}}
\newcommand{\bigO}{\mathcal{O}}
\newcommand{\Real}{\mathcal{Re}}
\newcommand{\poly}{\mathcal{P}}
\newcommand{\mat}{\mathcal{M}}
\DeclareMathOperator{\Span}{span}
\newcommand{\Hom}{\mathcal{L}}
\DeclareMathOperator{\Null}{null}
\DeclareMathOperator{\Range}{range}
\newcommand{\defeq}{\vcentcolon=}
\newcommand{\restr}[1]{|_{#1}}


\begin{document}

% ### Header - start ###
\begin{center}
    \hrule
    \vspace{0.4cm}
    { \textbf{{\large Homework 6}} \\ MATH 543 --- Linear Algebra}
\end{center}
{ Name: \textbf{Ali Zafari} \hspace{\fill} Spring 2023 } \newline\hrule
% ### Header - end ###

\section*{8.A Generalized Eigenvectors and Nilpotent Operators \xrfill[2pt]{3pt}[blue]}

\subsubsection*{Exercise 2}
\begin{itemize}
    \item \textbf{Eigenvalues.}
    \begin{align*}
        Tv=\lambda v\\
        T(v_1, v_2)-\lambda(v_1, v_2) = 0\\
        (-v_2 -\lambda v_1, v_1-\lambda v_2)=0\\
        \lambda = i, -i
    \end{align*}
    
    \item \textbf{Generalized Eigenspaces.}
    There is enough (2) distinct eigenvalues, so generalized eigenspaces are equal to eigenspaces.

    \begin{align*}
        E(i, T)=\Span(
        \left(
        \begin{array}{c}
          i\\
          1
        \end{array}
        \right)
        )
    \end{align*}

    \begin{align*}
        E(-i, T)=\Span(
        \left(
        \begin{array}{c}
          1\\
          i
        \end{array}
        \right)
        )
    \end{align*}
\end{itemize}
\vskip1mm\hrule

\subsubsection*{Exercise 3}
Let $n=\dim V$, by induction on $n$ we prove that $\Null(T-\lambda I)^n=\Null(T^{-1}-\frac{1}{\lambda}I)^n$.
\begin{itemize}
    \item $\mathbf{n=1}$. Let $v\in\Null(T-\lambda I)$ then $Tv=\lambda v$ or equivalently $T^{-1}v=\frac{1}{\lambda}v$ thus $v\in\Null(T^{-1}-\frac{1}{\lambda}I)$. For $w\in\Null(T^{-1}-\frac{1}{\lambda}I)$ same steps leads to $w\in\Null(T-\lambda I)$.
    \item \textbf{Let the hypothesis be correct $\forall i\quad 1\leq i \leq n-1$.}\\
    Let $v\in \Null(T-\lambda I)^n$:
    \begin{align*}
        (T-\lambda I)^nv=0 \Rightarrow (T-\lambda I)^{n-1}((T-\lambda I)v)=0 \Rightarrow (T-\lambda I)v\in \Null(T-\lambda I)^{n-1} 
    \end{align*}
    from induction hypothesis:
    \begin{align*}
        (T-\lambda I)v\in \Null(T^{-1}-\frac{1}{\lambda}I)^{n-1}
    \end{align*}
    so
    \begin{align*}
        (T^{-1}-\frac{1}{\lambda}I)^{n-1}(T-\lambda I)v=0 \Rightarrow (T-\lambda I)(T^{-1}-\frac{1}{\lambda}I)^{n-1}v=0
    \end{align*}
    thus
    \begin{align*}
        (T^{-1}-\frac{1}{\lambda}I)^{n-1}v\in\Null(T-\lambda I)=\Null(T^{-1}-\frac{1}{\lambda}I)
    \end{align*}
    and equivalently
    \begin{align*}
        (T^{-1}-\frac{1}{\lambda}I)(T^{-1}-\frac{1}{\lambda}I)^{n-1}v&=0\\
        (T^{-1}-\frac{1}{\lambda}I)^{n}v&=0\\
        v\in(T^{-1}-\frac{1}{\lambda}I)^{n}
    \end{align*}
    thus
    \begin{align*}
        \Null(T-\lambda I)^n\subseteq\Null(T^{-1}-\frac{1}{\lambda}I)^{n}
    \end{align*}
    the same steps can be followed to show the inclusion in the other direction and we have:
    \begin{align*}
        \Null(T-\lambda I)^n=\Null(T^{-1}-\frac{1}{\lambda}I)^n
    \end{align*}
\end{itemize}

\vskip1mm\hrule

\subsubsection*{Exercise 4}
Suppose $v\neq0$ and $v\in G(\alpha, T)\cap G(\beta, T)$. Since $v$ is an eigenvector for two generalized eigenspaces of distinct eigenvalues, it is a contradiction with 8.13. Therefore $v=0$.\\\\
({\color{teal}8.13: Generalized eigenvectors corresponding to distinct eigenvalues are linearly independent.})
\vskip1mm\hrule

% \subsubsection*{Exercise 10}
% \vskip1mm\hrule

% \subsubsection*{Exercise 13}
% \vskip1mm\hrule


\section*{8.B Decomposition of an Operator \xrfill[2pt]{3pt}[blue]}

\subsubsection*{Exercise 1}
$G(0, N)=\Null(N)^{\dim V}$ and since zero is the only eigenvalue $V=\Null(N)^{\dim V}$ or equivalently $N^{\dim V}=0 \quad\forall v\in V$. Thus $N $ is nilpotent. 
\vskip1mm\hrule

\subsubsection*{Exercise 2}
We consider a zero eigenvalue and a pair of complex conjugate eigenvalues:
\begin{align*}
    T(x,y,z)=(0,z,-y)
\end{align*}
\vskip1mm\hrule

\subsubsection*{Exercise 3}
Let $v\in V$ be eigenvector of $T$ corresponding to eigenvalue $\lambda$. There exists $u\in V$ such that $v=Su$. (S is surjective)
\begin{align*}
    S^{-1}TSu=S^{-1}Tv=S^{-1}(\lambda v)=\lambda u
\end{align*}
thus every eigenvalue for $T$ is also an eigenvalue for $S^{-1}TS$.
\vskip1mm\hrule

% \subsubsection*{Exercise 8}
% \vskip1mm\hrule

\subsubsection*{Exercise 10}
Look at LADW pp. 264.
$N^j=0$ for some $j\leq n$.\\
Since $N$ is normal, there exists a basis of $V$ consisting of orthonormal eigenvectors $e_1, \dots, e_n$
\\?
\vskip1mm\hrule

\section*{8.C Characteristic and Minimal Polynomials \xrfill[2pt]{3pt}[blue]}

\subsubsection*{Exercise 2}
For $T$
\begin{align*}
    1 \leq \dim G(5,T) \leq n-1\\
    1 \leq \dim G(6,T) \leq n-1
\end{align*}
therefore $(T-6I)^{n-1}(T-6I)^{n-1}$ is a multiple of characteristic polynomial $q(T)$ and thus equal to zero.
\vskip1mm\hrule

\subsubsection*{Exercise 4}
\begin{center}
$\chi_T(z)=(z-1)(z-5)^3$ and $p(z)=(z-1)(z-5)^2$
\end{center}
\begin{align*}
     A=
     \left(
     \begin{array}{cccc}
          1&0&0&0\\
          0&5&0&0\\
          0&0&5&1\\
          0&0&0&5
     \end{array}
     \right)
\end{align*}
\vskip1mm\hrule

\subsubsection*{Exercise 5}
\begin{center}
$\chi_T(z)=z(z-1)^2(z-3)=p(z)$
\end{center}
\begin{align*}
     A=
     \left(
     \begin{array}{cccc}
          0&0&0&0\\
          0&1&1&0\\
          0&0&1&0\\
          0&0&0&3
     \end{array}
     \right)
\end{align*}
\vskip1mm\hrule

\subsubsection*{Exercise 6}
\begin{center}
$\chi_T(z)=z(z-1)^2(z-3)$ and $p(z)=z(z-1)(z-3)$
\end{center}
\begin{align*}
     A=
     \left(
     \begin{array}{cccc}
          0&0&0&0\\
          0&1&0&0\\
          0&0&1&0\\
          0&0&0&3
     \end{array}
     \right)
\end{align*}
\vskip1mm\hrule

\subsubsection*{Exercise 8}
Let $\chi_T$ be the characteristic polynomial of $T$.
\begin{align*}
    \quad&T\text{ is invertible}\\
    \Longleftrightarrow\quad&\text{0 is not an eigenvlue of }T\\
    \Longleftrightarrow\quad&\chi_T(0)\neq0\\
    \Longleftrightarrow\quad&\text{constant term of }\chi_T(z) \text{ is not zero}
\end{align*}
\vskip1mm\hrule

\subsubsection*{Exercise 9}
We have $p(T)=4+5T-6T^2-7T^3+2T^4+T^5=0$.\\Then $T^{-5}p(T)=4T^{-5}+5T^{-4}-6T^{-3}-7T^{-2}+2T^{-1}+I=0$. Thus the minimal polynomial of $T^{-1}$ is $p(T^{-1})=T^{-5}+1.25T^{-4}-1.5T^{-3}-1.75T^{-2}+0.5T^{-1}+0.25I$
\vskip1mm\hrule

% \subsubsection*{Exercise 13}
% \vskip1mm\hrule

% \subsubsection*{Exercise 20}
% \vskip1mm\hrule


\section*{8.D Jordan Form \xrfill[2pt]{3pt}[blue]}

\subsubsection*{Exercise 1}
\begin{itemize}
    \item \textbf{Characteristic Polynomial.} Since $N$ has only zero eigenvalues and multiplicity of it is 4 then:
    \begin{align*}
        \chi_N(z)=z^4
    \end{align*}
    \item \textbf{Minimal Polynomial.} Since $N^3\neq0$, then the minimal polynomial is $p(N)=N^4$.
\end{itemize}
\vskip1mm\hrule

\subsubsection*{Exercise 2}
\begin{itemize}
    \item \textbf{Characteristic Polynomial.} Since $N$ has only zero eigenvalues and multiplicity of it is 6 then:
    \begin{align*}
        \chi_N(z)=z^6
    \end{align*}
    \item \textbf{Minimal Polynomial.} Since $N^3=0$, then the minimal polynomial is $p(N)=N^3$.
\end{itemize}
\vskip1mm\hrule

% \subsubsection*{Exercise 7}
% \vskip1mm\hrule

\section*{Not from Axler's Book. \xrfill[2pt]{3pt}[blue]}
{\color{violet}
Find a Jordan basis and the Jordan normal form for $A$.
\begin{align*}
     A=
     \left(
     \begin{array}{cccc}
          4&-4&-11&11\\
          3&-12&-42&42\\
          -2&12&37&-34\\
          -1&7&20&-17
     \end{array}
     \right)
\end{align*}
}
\begin{itemize}
    \item \textbf{Eigenvalues.}
    \begin{align*}
        det(A-\lambda I)&=0\\
        (\lambda-3)^4&=0\\
        \lambda&=3 \text{ with multiplicity of 4}
    \end{align*}

    \item \textbf{Number of blocks.} dimension of eigenspace $E(3, T)=\Null(T-3I)\\$ determines number of blocks in Jordan form.
    \begin{align*}
        \Null(A-3I)
        &=
        \Null\left(
        \begin{array}{cccc}
          1&-4&-11&11\\
          3&-15&-42&42\\
          -2&12&34&-34\\
          -1&7&20&-20
        \end{array}
        \right)\\
        &=
        \Null\left(
        \begin{array}{cccc}
          1&-4&-11&11\\
          0&-3&-9&9\\
          0&4&12&-12\\
          0&3&9&-9
        \end{array}
        \right) \quad\text{(row reduction)}\\
        &=
        \Null\left(
        \begin{array}{cccc}
          1&-4&-11&11\\
          0&-3&-9&9\\
          0&0&0&0\\
          0&0&0&0
        \end{array}
        \right) \quad\text{(row reduction)}
    \end{align*}
    thus the number of blocks is 2.
    \begin{align*}
        E(3, T)=\Span(
        \left(
        \begin{array}{c}
          1\\
          3\\
          0\\
          1
        \end{array}
        \right)
        ,
        \left(
        \begin{array}{c}
          -1\\
          -3\\
          1\\
          0
        \end{array}
        \right)
        )
    \end{align*}
    \item \textbf{Jordan Form.} thus the Jordan form is one of the below:
    \begin{align*}
     \left(
     \begin{array}{cccc}
          \cellcolor{lightgray}3&0&0&0\\
          0&\cellcolor{lightgray}3&\cellcolor{lightgray}1&\cellcolor{lightgray}0\\
          0&\cellcolor{lightgray}0&\cellcolor{lightgray}3&\cellcolor{lightgray}1\\
          0&\cellcolor{lightgray}0&\cellcolor{lightgray}0&\cellcolor{lightgray}3
     \end{array}
     \right)
    \text{ or }
    \left(
     \begin{array}{cccc}
          \cellcolor{lightgray}3&\cellcolor{lightgray}1&0&0\\
          \cellcolor{lightgray}0&\cellcolor{lightgray}3&0&0\\
          0&0&\cellcolor{lightgray} 3&\cellcolor{lightgray}1\\
          0&0&\cellcolor{lightgray}0&\cellcolor{lightgray}3
     \end{array}
     \right)
\end{align*}
    \item \textbf{Generalized Eigenspace.}
    \begin{align*}
        \Null(A-3I)^2
        &=
        \Null\left(
        \begin{array}{cccc}
          0&1&3&-3\\
          0&3&9&-9\\
          0&-2&-6&6\\
          0&-1&-3&3
        \end{array}
        \right)\\
        &=
        \Null\left(
        \begin{array}{cccc}
          0&1&3&-3\\
          0&0&0&0\\
          0&0&0&0\\
          0&0&0&0
        \end{array}
        \right) \quad\text{(row reduction)}
    \end{align*}
    Since $\dim\Null(A-3I)^2=3<4$, we continue for $\Null(A-3I)^3$:
    \begin{align*}
        \Null(A-3I)^3
        &=
        \Null\left(
        \begin{array}{cccc}
          0&0&0&0\\
          0&0&0&0\\
          0&0&0&0\\
          0&0&0&0
        \end{array}
        \right)
    \end{align*}
    Thus $\dim\Null(A-3I)^3=4$.


    Therefore Jordan Normal form consists of two blocks of 1x1 and 3x3 as below:
    \begin{align*}
     J=\left(
     \begin{array}{cccc}
          \cellcolor{lightgray}3&0&0&0\\
          0&\cellcolor{lightgray}3&\cellcolor{lightgray}1&\cellcolor{lightgray}0\\
          0&\cellcolor{lightgray}0&\cellcolor{lightgray}3&\cellcolor{lightgray}1\\
          0&\cellcolor{lightgray}0&\cellcolor{lightgray}0&\cellcolor{lightgray}3
     \end{array}
     \right)
     \tag{$\clubsuit$}
     \label{jordan}
     \end{align*}
    \item \textbf{Jordan Basis.} To choose the first eigenvector, we look for $v_1\neq0$ such that $(A-3I)^3v_1=0$ and $(A-3I)^2v_1\neq0$.
    Then we choose $v_4\neq0$ such that $(A-3I)v_4=0$.
    \begin{align*}
        &v_1
        =
        \left(
        \begin{array}{c}
          0\\
          1\\
          0\\
          0
        \end{array}
        \right)\\
        &v_2=(A-3I)v_1=\left(
        \begin{array}{c}
          -4\\
          -15\\
          12\\
          7
        \end{array}
        \right)\\
        &v_3=(A-3I)^2v_1=\left(
        \begin{array}{c}
          1\\
          3\\
          -2\\
          -1
        \end{array}
        \right)\\
        &v_4=\left(
        \begin{array}{c}
          1\\
          3\\
          0\\
          1
        \end{array}
        \right)\\
    \end{align*}
    To have exactly the Jordan form as in (\ref{jordan}), the order of the elements in matrix P below matters and should be exactly like $[\quad v_4\quad|\quad(A-3I)^2v_1\quad|\quad(A-3I)v_1\quad|\quad v_1\quad]$, i.e.,
    \begin{align*}
        P=\left(
        \begin{array}{cccc}
          1&1&-4&0\\
          3&3&-15&1\\
          0&-2&12&0\\
          1&-1&7&0
        \end{array}
        \right)
    \end{align*}
    Then we have for sure $J = P^{-1}AP$ with $J$ as showed in (\ref{jordan}).
\end{itemize}
\end{document}

