\documentclass[12pt, letterpaper]{scrartcl}


\usepackage{fullpage} % Set margins and place page numbers at bottom center
\usepackage[shortlabels]{enumitem} % Use a. in the enumerate
\usepackage{amsmath} % aligned equations
\usepackage{graphicx} % include figure
\usepackage{float} % usage of H for figure float
\usepackage{amssymb} % \blacksqure
\usepackage{xhfill} % fill horizontal line

\usepackage{xcolor} % colors
\usepackage{sectsty} % section coloring
\sectionfont{\color{blue}}  % sets colour of sections

%%%%%%%%%%%%%%%%%%%%%%%%%%%%%%%%%%%%%%%%%%%%%%%%%%%%%%%%%%%%%%%%%%%%%%
% MY COMMANDS                                                        %
%%%%%%%%%%%%%%%%%%%%%%%%%%%%%%%%%%%%%%%%%%%%%%%%%%%%%%%%%%%%%%%%%%%%%%
\newcommand{\Z}{\mathbb{Z}}
\newcommand{\R}{\mathbb{R}}
\newcommand{\C}{\mathbb{C}}
\newcommand{\F}{\mathbb{F}}
\newcommand{\bigO}{\mathcal{O}}
\newcommand{\Real}{\mathcal{Re}}
\newcommand{\poly}{\mathcal{P}}
\newcommand{\mat}{\mathcal{M}}
\DeclareMathOperator{\Span}{span}
\newcommand{\Hom}{\mathcal{L}}
\DeclareMathOperator{\Null}{null}
\DeclareMathOperator{\Range}{range}
\newcommand{\defeq}{\vcentcolon=}
\newcommand{\restr}[1]{|_{#1}}


\begin{document}

% ### Header - start ###
\begin{center}
    \hrule
    \vspace{0.4cm}
    { \textbf{{\large Extra Problems}} \\ MATH 543 --- Linear Algebra}
\end{center}
{ Name: \textbf{Ali Zafari} \hspace{\fill} Spring 2023 } \newline\hrule
% ### Header - end ###

% \section*{5.B Eigenvectors and Upper-Triangular Matrices \xrfill[2pt]{3pt}[blue]}
\subsubsection*{}
{\color{blue}Let $\dim V=n$. If  $T\in\mathcal{L}(V)$ is nilpotent, then $T^n=0$.}
\\\\
$T^j=0$ for some integer $j\geq 1$  thus $\Null T^j=V$.\\
Growth of $\Null T^k$ with increase of $k$ (non-negative integer) stops at $k=n$.
\begin{itemize}
    \item $\mathbf{j<n}$:  since $\Null T^j\subseteq \Null T^n$ then $\Null T^n=V$. 

    \item $\mathbf{j\geq n}$: having $\Null T^n\neq V$ is a contradiction. 
\end{itemize}
therefore $\Null T^n=V$ so $T^n=0$.
\vskip1mm\hrule


\subsubsection*{}
{\color{blue}If $\lambda$ is an eigenvalue for $T\in\mathcal{L}(V)$ then $\lambda^n$ is an eigenvalue for $T^n$.}
\\\\
For $v\neq0$ we have $Tv=\lambda v$:
\begin{align*}
    T^2v&=T(Tv)=T(\lambda v)=\lambda Tv=\lambda^2v\\
    T^3v&=T(T^2v)=T(\lambda^2v)=\lambda^2 Tv=\lambda^3v\\
    &\vdots\\
    T^nv&=T(T^{n-1}v)=T(\lambda^{n-1}v)=\lambda^{n-1} Tv=\lambda^nv
\end{align*}
thus $\lambda^n$ is an eigenvalue for $T^n$.
\vskip1mm\hrule


\subsubsection*{}
{\color{blue}If $\lambda^n$ is an eigenvalue for $T^n\in\mathcal{L}(V)$ then $\lambda$ is an eigenvalue for $T$?}
\\\\
For $v\neq0$ we have $T^nv=\lambda^n v$:
\begin{align*}
    (T^n-\lambda^nI)v&=0\\
    p(T)v&=0
\end{align*}
 where $p(z)=z^n-\lambda^n \in \mathcal{P}(\C)$.\\
 We use factorization $p(z)=(z-\lambda)(z^{n-1}+z^{n-2}\lambda+\dots+z\lambda^{n-2}+\lambda^{n-1})$:
\begin{align*}
    p(T)v=(T-\lambda)(T^{n-1}+\lambda T^{n-2}+\dots+\lambda^{n-2}T+\lambda^{n-1})v=0
\end{align*}
thus if $(T^{n-1}+\lambda T^{n-2}+\dots+\lambda^{n-2}T+\lambda^{n-1})v\neq0$, then $\lambda$ must be an eigenvalue for $T$.
\vskip1mm\hrule

\end{document}

