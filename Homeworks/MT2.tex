\documentclass[12pt, letterpaper]{scrartcl}


\usepackage{fullpage} % Set margins and place page numbers at bottom center
\usepackage[shortlabels]{enumitem} % Use a. in the enumerate
\usepackage{amsmath} % aligned equations
\usepackage{graphicx} % include figure
\usepackage{float} % usage of H for figure float
\usepackage{amssymb} % \blacksqure
\usepackage{xhfill} % fill horizontal line

\usepackage{xcolor} % colors
\usepackage{sectsty} % section coloring
\sectionfont{\color{blue}}  % sets colour of sections

%%%%%%%%%%%%%%%%%%%%%%%%%%%%%%%%%%%%%%%%%%%%%%%%%%%%%%%%%%%%%%%%%%%%%%
% MY COMMANDS                                                        %
%%%%%%%%%%%%%%%%%%%%%%%%%%%%%%%%%%%%%%%%%%%%%%%%%%%%%%%%%%%%%%%%%%%%%%
\newcommand{\Z}{\mathbb{Z}}
\newcommand{\R}{\mathbb{R}}
\newcommand{\C}{\mathbb{C}}
\newcommand{\F}{\mathbb{F}}
\newcommand{\bigO}{\mathcal{O}}
\newcommand{\Real}{\mathcal{Re}}
\newcommand{\poly}{\mathcal{P}}
\newcommand{\mat}{\mathcal{M}}
\DeclareMathOperator{\Span}{span}
\newcommand{\Hom}{\mathcal{L}}
\DeclareMathOperator{\Null}{null}
\DeclareMathOperator{\Range}{range}
\newcommand{\defeq}{\vcentcolon=}
\newcommand{\restr}[1]{|_{#1}}


\begin{document}

% ### Header - start ###
\begin{center}
    \hrule
    \vspace{0.4cm}
    { \textbf{{\large Midterm 2}} \\ MATH 543 --- Linear Algebra}
\end{center}
{ Name: \textbf{Ali Zafari} \hspace{\fill} Spring 2023 } \newline\hrule
% ### Header - end ###

% \section*{5.B Eigenvectors and Upper-Triangular Matrices \xrfill[2pt]{3pt}[blue]}
\subsubsection*{Problem 1}
\begin{enumerate}[(a)]
    \item $\mathcal{M}(T)=A$.\\\\
    \textbf{Eigenvalues}:
     \begin{align*}
         \det (A-\lambda I)=
         \left|
         \begin{array}{ccc}
              -\frac{1}{2}-\lambda&0&-\frac{3}{2}\\
              -\frac{3}{2}&-\lambda&-\frac{3}{2}\\
              0&0&1-\lambda
         \end{array}
         \right|
         =(-\frac{1}{2}-\lambda)(-\lambda)(1-\lambda)
         =0
     \end{align*}
     thus eigenvalues are $\lambda_1=0$, $\lambda_2=-\frac{1}{2}$ and $\lambda_3=1$.\\
     
     \textbf{Eigenspaces and Eigenvectors}:
     \begin{itemize}
         \item $E(0,T)=\Null(T-0I)$=?
             \begin{align*}
                 \left[
                 \begin{array}{ccc}
                      -\frac{1}{2}&0&-\frac{3}{2}\\
                      -\frac{3}{2}&0&-\frac{3}{2}\\
                      0&0&1
                 \end{array}
                 \right]
                 \left[
                 \begin{array}{c}
                      x\\
                      y\\
                      z
                 \end{array}
                 \right]
                 =
                 \left[
                 \begin{array}{c}
                      0\\
                      0\\
                      0
                 \end{array}
                 \right]
                 \Longrightarrow
                 E(0,T)=span((0,1,0))
                 \Longrightarrow
                 v_1=(0,1,0)
             \end{align*}
        \item $E(-\frac{1}{2},T)=\Null(T+\frac{1}{2}I)$=?
             \begin{align*}
                 \left[
                 \begin{array}{ccc}
                      0&0&-\frac{3}{2}\\
                      -\frac{3}{2}&\frac{1}{2}&-\frac{3}{2}\\
                      0&0&\frac{3}{2}
                 \end{array}
                 \right]
                 \left[
                 \begin{array}{c}
                      x\\
                      y\\
                      z
                 \end{array}
                 \right]
                 =
                 \left[
                 \begin{array}{c}
                      0\\
                      0\\
                      0
                 \end{array}
                 \right]
                 \Longrightarrow
                 E(-\frac{1}{2},T)=span((1,-3,0))
                 \Longrightarrow
                 v_2=(1,-3,0)
             \end{align*}

        \item $E(1,T)=\Null(T-1I)$=?
             \begin{align*}
                 \left[
                 \begin{array}{ccc}
                      -\frac{3}{2}&0&-\frac{3}{2}\\
                      -\frac{3}{2}&-1&-\frac{3}{2}\\
                      0&0&0
                 \end{array}
                 \right]
                 \left[
                 \begin{array}{c}
                      x\\
                      y\\
                      z
                 \end{array}
                 \right]
                 =
                 \left[
                 \begin{array}{c}
                      0\\
                      0\\
                      0
                 \end{array}
                 \right]
                 \Longrightarrow
                 E(1,T)=span((1,0,-1))
                 \Longrightarrow
                 v_3=(1,0,-1)
             \end{align*}
     \end{itemize}
     Since $\dim V=\Sigma_{i=1}^3\dim E(\lambda_i,T)=3$ then $A$ is diagonalizable.
     
     \item 
     Since $A$ is diagonalizable, the equality $D=S^{-1}AS$ holds when $S$ consists of the eigenvectors, and $D$ is diagonal matrix with corresponding eigenvalues. Thus:
     \begin{align*}
         \lim_{n\rightarrow\infty}A^n=\lim_{n\rightarrow\infty}SD^nS^{-1}&=
         \lim_{n\rightarrow\infty}\left[
         \begin{array}{ccc}
              0&1&1\\
              1&-3&0\\
              0&0&-1
         \end{array}
         \right]
         \left[
         \begin{array}{ccc}
              0&0&0\\
              0&-\frac{1}{2}&0\\
              0&0&1
         \end{array}
         \right]^n
         \left[
         \begin{array}{ccc}
              0&1&1\\
              1&-3&0\\
              0&0&-1
         \end{array}
         \right]^{-1}\\
         &=\lim_{n\rightarrow\infty}\left[
         \begin{array}{ccc}
              0&1&1\\
              1&-3&0\\
              0&0&-1
         \end{array}
         \right]
         \left[
         \begin{array}{ccc}
              0&0&0\\
              0&(-\frac{1}{2})^n&0\\
              0&0&1^n
         \end{array}
         \right]
         \left[
         \begin{array}{ccc}
              3&1&3\\
              1&0&1\\
              0&0&-1
         \end{array}
         \right]\\
         &=\left[
         \begin{array}{ccc}
              0&1&1\\
              1&-3&0\\
              0&0&-1
         \end{array}
         \right]
         \left[
         \begin{array}{ccc}
              0&0&0\\
              0&0&0\\
              0&0&1
         \end{array}
         \right]
         \left[
         \begin{array}{ccc}
              3&1&3\\
              1&0&1\\
              0&0&-1
         \end{array}
         \right]\\
         &=
         \left[
         \begin{array}{ccc}
              0&0&-1\\
              0&0&0\\
              0&0&1
         \end{array}
         \right]
     \end{align*}
\end{enumerate}
 
\vskip1mm\hrule


\subsubsection*{Problem 2}
The only distinct eigenvalue of $T\in \mathcal{L}(\C^n)$ is $\lambda=0$.
\begin{itemize}
    \item[$\Longrightarrow$]\mbox{}\\
    If $T$ is diagonalizable then:\\\\
    $\color{darkgray}\dim \C^n=\dim E(0,T)=\dim\Null(T-0I)=\dim\Null T $ therefore $\color{darkgray}\dim\Null T=n$.
    
    Since $\color{darkgray}\Null T \subseteq \C^n$, we have $\color{darkgray}\Null T=\C^n$. Therefore $\color{darkgray}\forall v\in \C^n \quad Tv=0$ meaning that $\color{darkgray}T=0$. 

    \item[$\Longleftarrow$]\mbox{}\\
    If $T=0$ then:\\\\
    $\color{darkgray}E(0,T)=\Null(T-0I)=\Null T=\{v | Tv=0, v\in\C^n\}=\C^n$.
    
    Since $\color{darkgray}\dim \C^n=\dim E(0,T)=n$ then $\color{darkgray}T$ is diagonalizable.
\end{itemize}
\vskip1mm\hrule


\subsubsection*{Problem 3}
\begin{enumerate}[(a)]
    \item $v_1=(-1,0,1)$ and $v_2=(0,1,-1)$, using Gram-Schmidt procedure: 
    \begin{align*}
        e_1=\frac{v_1}{\Vert v_1 \Vert}=\frac{(-1,0,1)}{\sqrt{2}}=\frac{1}{\sqrt{2}}(-1,0,1)
    \end{align*}
    and
    \begin{align*}
        e_2&=\frac{v_2-\langle v_2,e_1\rangle e_1}{\Vert v_2-\langle v_2,e_1\rangle e_1\Vert}\\
        &=\frac{(0,1,-1)-(-\frac{1}{\sqrt{2}})\frac{1}{\sqrt{2}}(-1,0,1)}{\Vert(0,1,-1)-(-\frac{1}{\sqrt{2}})\frac{1}{\sqrt{2}}(-1,0,1)\Vert}\\
        &=\sqrt{\frac{2}{3}}(-\frac{1}{2},1,-\frac{1}{2})
    \end{align*}

    \item 
    Since the origin is in $U$, the minimum distance to it from $U$ (length of orthogonal projection $P_U\mathbf{0}$) is 0.
    \begin{align*}
        P_U\mathbf{0}=\langle(0,0,0),e_1\rangle e_1 + \langle(0,0,0),e_2\rangle e_2=(0,0,0)
    \end{align*}
\end{enumerate}
\vskip1mm\hrule


\subsubsection*{Problem 4}
Let $M,N,P,Q\in \R^2$ denote the vertices of the MNPQ quadrilateral.
\begin{itemize}
    \item[$\Longrightarrow$]\mbox{}\\If $\mathbf{PM\perp NQ}$, then:\\\\
    Diagonals can be written as follows:
    \begin{align*}
        PM=M-P\\
        NQ=Q-N
    \end{align*}
    Then:
    \begin{align*}
        \langle M-P,Q-N\rangle&=0\\
        \langle M,Q\rangle-\langle M,N\rangle-\langle P,Q\rangle+\langle P,N\rangle&=0\\
        2\langle M,Q\rangle-2\langle M,N\rangle-2\langle P,Q\rangle+2\langle P,N\rangle&=0\\
        \left\{
        \begin{array}{cccccccc}
        &2\langle M,Q\rangle&-&2\langle M,N\rangle&-&2\langle P,Q\rangle&+&2\langle P,N\rangle\\
        +&{\color{blue}\langle N,N\rangle}&-&{\color{blue}\langle N,N\rangle}&+&{\color{brown}\langle M,M\rangle}&-&{\color{brown}\langle M,M\rangle}\\
        +&{\color{gray}\langle Q,Q\rangle}&-&{\color{gray}\langle Q,Q\rangle}&+&{\color{teal}\langle P,P\rangle}&-&{\color{teal}\langle P,P\rangle}
        \end{array}
        \right\}&=0\\
        \left\{
        \begin{array}{cccccccc}
        &{\color{blue}\langle N,N\rangle}&-&\langle N,M\rangle&-&\langle M,N\rangle&+&{\color{brown}\langle M,M\rangle}\\
        +&{\color{gray}\langle Q,Q\rangle}&-&\langle Q,P\rangle&-&\langle P,Q\rangle&+&{\color{teal}\langle P,P\rangle}\\
        -&{\color{teal}\langle P,P\rangle}&+&\langle P,N\rangle&+&\langle N,P\rangle&-&{\color{blue}\langle N,N\rangle}\\
        -&{\color{gray}\langle Q,Q\rangle}&+&\langle Q,M\rangle&+&\langle M,Q\rangle&-&{\color{brown}\langle M,M\rangle}
        \end{array}
        \right\}&=0 \quad \text{(re-arranging)}\\
        \left\{
        \begin{array}{cc}
        &\langle N-M, N-M\rangle\\
        +&\langle Q-P, Q-P\rangle\\
        -&\langle P-N, P-N\rangle\\
        -&\langle Q-M, Q-M\rangle
        \end{array}
        \right\}&=0%\\
        % \langle N-M, N-M\rangle+\langle Q-P, Q-P\rangle - \langle P-N, P-N\rangle-\langle M-Q, M-Q\rangle &= 0
    \end{align*}
    Thus:
    \begin{align*}
        \langle N-M, N-M\rangle+\langle Q-P, Q-P\rangle &= \langle P-N, P-N\rangle+\langle Q-M, Q-M\rangle\\
        |MN|^2+|PQ|^2&=|NP|^2+|MQ|^2
    \end{align*}
    
    \item[$\Longleftarrow$]\mbox{}\\If $\mathbf{|MN|^2+|PQ|^2=|NP|^2+|MQ|^2}$, then:\\\\
    Let the sides of the quadrilateral be written in terms of its vertices as $MN=N-M$, $PQ=Q-P$, $NP=P-N$ and $MQ=Q-M$ then:
    \begin{align*}
        &\langle N-M, N-M\rangle+\langle Q-P, Q-P\rangle = \langle P-N, P-N\rangle+\langle Q-M, Q-M\rangle\\
        &\langle N-M, N-M\rangle+\langle Q-P, Q-P\rangle - \langle P-N, P-N\rangle-\langle Q-M, Q-M\rangle = 0
    \end{align*}
    \begin{align*}
        \left\{
        \begin{array}{cccccccc}
        &\langle N,N\rangle&-&\langle N,M\rangle&-&\langle M,N\rangle&+&\langle M,M\rangle\\
        &\langle Q,Q\rangle&-&\langle Q,P\rangle&-&\langle P,Q\rangle&+&\langle P,P\rangle\\
        -&\langle P,P\rangle&+&\langle P,N\rangle&+&\langle N,P\rangle&-&\langle N,N\rangle\\
        -&\langle Q,Q\rangle&+&\langle Q,M\rangle&+&\langle M,Q\rangle&-&\langle M,M\rangle
        \end{array}
        \right\}&=0\\
        -2\langle N,M\rangle-2\langle Q,P\rangle+2\langle P,N\rangle+2\langle M,Q\rangle&=0\\
        -\langle M,N\rangle-\langle P,Q\rangle+\langle P,N\rangle+\langle M,Q\rangle&=0\\
        \langle M, Q-N\rangle - \langle P, Q-N\rangle&=0\\
        \langle M-P, Q-N\rangle&=0\\
        PM\perp NQ
    \end{align*}    
    
\end{itemize}
\vskip1mm\hrule


\end{document}

