\documentclass[12pt, letterpaper]{scrartcl}


\usepackage{fullpage} % Set margins and place page numbers at bottom center
\usepackage[shortlabels]{enumitem} % Use a. in the enumerate
\usepackage{amsmath} % aligned equations
\usepackage{graphicx} % include figure
\usepackage{float} % usage of H for figure float
\usepackage{amssymb} % \blacksqure
\usepackage{xhfill} % fill horizontal line

\usepackage{xcolor} % colors
\usepackage{sectsty} % section coloring
\sectionfont{\color{blue}}  % sets colour of sections

%%%%%%%%%%%%%%%%%%%%%%%%%%%%%%%%%%%%%%%%%%%%%%%%%%%%%%%%%%%%%%%%%%%%%%
% MY COMMANDS                                                        %
%%%%%%%%%%%%%%%%%%%%%%%%%%%%%%%%%%%%%%%%%%%%%%%%%%%%%%%%%%%%%%%%%%%%%%
\newcommand{\Z}{\mathbb{Z}}
\newcommand{\R}{\mathbb{R}}
\newcommand{\C}{\mathbb{C}}
\newcommand{\F}{\mathbb{F}}
\newcommand{\bigO}{\mathcal{O}}
\newcommand{\Real}{\mathcal{Re}}
\newcommand{\poly}{\mathcal{P}}
\newcommand{\mat}{\mathcal{M}}
\DeclareMathOperator{\Span}{span}
\newcommand{\Hom}{\mathcal{L}}
\DeclareMathOperator{\Null}{null}
\DeclareMathOperator{\Range}{range}
\newcommand{\defeq}{\vcentcolon=}
\newcommand{\restr}[1]{|_{#1}}


\begin{document}

% ### Header - start ###
\begin{center}
    \hrule
    \vspace{0.4cm}
    { \textbf{{\large Homework 3}} \\ MATH 543 --- Linear Algebra}
\end{center}
{ Name: \textbf{Ali Zafari} \hspace{\fill} Spring 2023 } \newline\hrule
% ### Header - end ###

\section*{3.D Invertibility and Isomorphic Vector Spaces \xrfill[2pt]{3pt}[blue]}
\subsubsection*{Exercise 7}
\begin{enumerate}[(a)]
    \item 
    Three conditions for $E$ to be a subspace $\mathcal{L}(V,W)$:
    \begin{enumerate}[1.]
        \item \textbf{additive identity} Obviously linear map $0\in \mathcal{L}(V,W)$ is a member of $E$.
        \item \textbf{closed under addition} Let $T_1,T_2\in E$:
        \begin{align*}
            (T_1+T_2)v=T_1v+T_2v=0+0=0
        \end{align*}
        therefore $T_1+T_2\in E$.
        \item \textbf{closed under scalar multiplication} Let $T\in E$ and $\lambda\in\F$
        \begin{align*}
            (\lambda T)v=\lambda Tv=\lambda 0=0
        \end{align*}
        therefore $\lambda T\in E$.
    \end{enumerate}

    \item 
    Let's extend $v$ to a basis $v, v_2, \dots, v_n$. And assume $w_1, \dots, w_m$ is a basis for $W$. Then, $\mathcal{M}$ is an isomorphism between $\mathcal{L}(V,W)$ and $\F^{m,n}$. Having $Tv=0$ is equivalent to have first column of $\mathcal{M}$ be zero. Therefore $\dim E=m(n-1)$.
    
\end{enumerate}
\vskip1mm\hrule

\subsubsection*{Exercise 10}
\begin{itemize}
    \item[$\Longrightarrow$]\mbox{}\\
    \begin{align*}
        ST=I \Longrightarrow TS=I
    \end{align*}
    Let $v\in V$ then $Tv=u\in V$:
    \begin{align*}
        Tv&=u\\
        STv&=Su\\
        Iv&=Su\\
        v&=Su\\
        Tv&=TSu\\
        u&=TSu\\
    \end{align*}
    therefore $TS=I$.
    
    \item[$\Longleftarrow$]\mbox{}\\
    \begin{align*}
        TS=I \Longrightarrow ST=I
    \end{align*}
    Let $v\in V$ then $Sv=u\in V$:
    \begin{align*}
        Sv&=u\\
        TSv&=Tu\\
        Iv&=Tu\\
        v&=Tu\\
        Sv&=STu\\
        u&=STu\\
    \end{align*}
    therefore $ST=I$.

\end{itemize}
\vskip1mm\hrule

\subsubsection*{Exercise 19}
\begin{enumerate}[(a)]
    \item 

    Assume $p\in \mathcal{P}_n(\R)$ is of degree $n$. Then arbitrary $T:\mathcal{P}(\R)\rightarrow\mathcal{P}(\R)$ can be thought as $T_n:\mathcal{P}_n(\R)\rightarrow\mathcal{P}_n(\R)$ due to $\deg Tp\leq \deg p$.\\$\mathcal{P}_n(\R)$ is finite dimensional and $T_n$ is an operator on it, so injectivity of $T_n$ implies its surjectivity.
    
    \item \emph{proof by contradiction.}\\Let $\deg Tp \neq \deg p$ for \textbf{invertible} operator $T\in\mathcal{L}(\mathcal{P}(\R))$. Then $\deg Tp \leq \deg p$ translates to $\deg Tp < \deg p$.\\But any linear map to a smaller dimensional space is \textbf{not injective} (\emph{contradiction}). Therefore we must have $\deg Tp = \deg p$.
\end{enumerate}
\vskip1mm\hrule

\clearpage
\section*{3.E Products and Quotients of Vector Spaces \xrfill[2pt]{3pt}[blue]}
\subsubsection*{Exercise 2}
Since $V_1\times\dots\times V_m$ is finite-dimensional it has a basis of length
\begin{align*}
    \dim V_1+\dots+\dim V_m&<\infty\\
    \dim V_i&<\infty \quad (\dim V_i\geq 0) \quad\forall i  
\end{align*}
\vskip1mm\hrule

\subsubsection*{Exercise 8}
\begin{itemize}
    \item[$\Longrightarrow$]\mbox{}\\
    A is an affine subset of V. Then exists subspace $U$ of $V$ such that $A=a+U$ where $a\in V$.

    For $v,w\in A$ there exists $u_1,u_2\in U$ such that $v=a+u_1$ and $w=a+u_2$.

    $\forall \lambda \in \F$:
    \begin{align*}
        \lambda v+(1-\lambda)w=\lambda(a+u_1)+(1-\lambda)(a+u_2)=\underbrace{a+\underbrace{\lambda u_1+(1-\lambda)u_2}_{\in U}}_{\in A}
    \end{align*}
    
    \item[$\Longleftarrow$]\mbox{}\\
    $\lambda v+(1-\lambda)w\in A \quad \forall v,w\in A , \forall\lambda\in\F$ by choosing $a\in A$ we define $U\triangleq-a+A$.\\
    
    For $U$ to be a subspace of $V$:
    \begin{enumerate}
        \item \textbf{additive identity}. 
        
        Obviously $-a\in A$, so $0\in U$.
        
        \item \textbf{closed scalar multiplication}.

        Let $u\in U$ then exists $b\in A$ such that $u=-a+b$:
        \begin{align*}
            \lambda b + (1-\lambda)a &\in A\\
            a + \lambda(-a+b)
            %\underbrace{2[\frac{1}{2}\underbrace{(2-\lambda )a}_{\in A}+(1-\frac{1}{2})\underbrace{\lambda b)}_{\in A}]}_{\in A}
            &\in A\\
            \lambda(\underbrace{-a+b}_{=u})&\in \underbrace{-a+A}_{= U}
        \end{align*}
        
        \item \textbf{closed addition}.
        
        Let $u_1, u_2\in U$, then exist $a_1, a_2\in A$ such that $u_1=-a+a_1$ and $u_2=-a+a_2$.
        \begin{align*}
            u_1+u_2=-2a+a_1+a_2=2(-a+\frac{1}{2}a_1+\frac{1}{2}a_2)=\underbrace{2(\underbrace{-a+\underbrace{\frac{1}{2}a_1+(1-\frac{1}{2})a_2}_{\in A}}_{\in U})}_{\in U \text{  (closed scalar mult.)}}
        \end{align*}
        
    \end{enumerate}
    
\end{itemize}
\vskip1mm\hrule

\subsubsection*{Exercise 11}
\begin{enumerate}[(a)]
    \item 
    Let $v,w\in A$
    \begin{align*}
        v&=a_1v_1+\dots+a_mv_m\\
        w&=b_1v_1+\dots+b_mv_m
    \end{align*}
    where $a_1,\dots,a_m,b_1,\dots,b_m\in\F$ and $\sum_{i=1}^m a_i =\sum_{i=1}^m b_i=1$.
    
    $\forall \lambda \in \F$:
    \begin{align*}
        \lambda v+(1-\lambda)w&=\lambda(a_1v_1+\dots+a_mv_m)+(1-\lambda)(b_1v_1+\dots+b_mv_m)\\
        &=\underbrace{(\lambda a_1+(1-\lambda)b_1) v_1 + \dots + (\lambda a_m+(1-\lambda)b_m)v_m}_{\in A \text{ (by definition of $A$, since } \sum_{i=1}^m\lambda a_i+(1-\lambda)b_i=1)}
    \end{align*}
    By using the result of \textbf{Exercise 8}, $A$ will be an affine subset of $V$.
    \item Let $v\in V$ and $U$ is a subspace of $V$, such that $v_1,\dots,v_m\in v+U$. Therefore exists $u_1, \dots, u_m\in U$ such that $v_i=v+u_i \quad\forall i=1,\dots,n$.\\\\
    Assume $\lambda_1,\dots, \lambda_m\in\F$ such that $\sum_{i=1}^m\lambda_i=1$, then
    \begin{align*}
    \underbrace{\lambda_1v_1+\dots+\lambda_mv_m}_{\text{arbitrary element in }A}&=\lambda_1(v+u_1)+\dots+\lambda_m(v+u_m)\\
    &=(\lambda_1+\dots+\lambda_m)v + \lambda_1u_1+\dots+\lambda_mu_m\\
    &=\underbrace{v+\underbrace{\lambda_1u_1+\dots+\lambda_mu_m}_{\in U}}_{\in v+U}
    \end{align*}
    Therefore $A\subset v+U$.
    
    \item 
    Let $v\in A$:
    \begin{align*}
        v&=a_1v_1+\dots+a_mv_m=(1-\sum_{i=2}^m a_i)v_1+a_2v_2+\dots+a_mv_m\\
        &=v_1+a_2(v_2-v_1)+a_3(v_3-v_1)+\dots+a_m(v_m-v_1)
    \end{align*}
    %vectors $v_2-v_1, \dots, v_m-v_1$ are linearly independent and a basis for defined space 
    Let $U\triangleq span(v_2-v_1, \dots, v_m-v_1)$. Therefore $v=v_1+U$. As a result $A\subset v_1+U$.

    Now suppose $w\in v_1+U$ and $b_2, \dots, b_m\in \F$
    \begin{align*}
        w&=v_1+b_2(v_2-v_1)+\dots+b_m(v_m-v_1)\\
        &=(1-\sum_{i=2}^m b_i)v_1+b_2v_2+\dots+b_mv_m
    \end{align*}
    since $1-\sum_{i=2}^m b_i+ b_2+\dots+b_m=1$, then $w\in A$ meaning that $v_1+U\subset A$.

    As a result $A=v_1+U$ and obviously $\dim U\leq m-1$.
\end{enumerate}
\vskip1mm\hrule

\subsubsection*{Exercise 12}
$V/U$ has basis of $v_1+U, \dots, v_n+U$. $\forall v\in V \quad \exists a_1, \dots,a_n\in\F$ such that
\begin{align*}
    v+U=\sum_{i=1}^n a_i(v_i+U)\Rightarrow v-\sum_{i=1}^n a_iv_i\in U
\end{align*}
Let's define $T:V\rightarrow U\times V/U$ then $Tv=(v-\sum_{i=1}^n a_iv_i, \sum_{i=1}^n a_i(v_i+U))\quad\forall v\in V$.\\

We will show $T$ is an isomorphism.
\begin{itemize}
    \item \textbf{linearity}.
    $\forall x,y \in V$:
    \begin{align*}
        x+U=\sum_{i=1}^n b_i(v_i+U)\\
        y+U = \sum_{i=1}^n c_i(v_i+U)
    \end{align*}
    where $b_1, \dots, b_n, c_1, \dots, c_n\in\F$. Their linear combination is $d_1(x+U)+d_2(y+U)=\sum_{i=1}^n (d_1b_i+d_2c_i)(v_i+U)$ where $d_1,d_2\in\F$.

    So
    \begin{align*}
        T(d_1x+d_2y)&=(d_1x+d_2y-\sum_{i=1}^n (d_1b_i+d_2c_i)v_i, \sum_{i=1}^n (d_1b_i+d_2c_i)(v_i+U))\\
        &=d_1Tx+d_2Ty
    \end{align*}
    
    
    \item \textbf{injectivity}.

    If $Tv=0$, then $v=0$, as shown below
    \begin{align*}
        Tv=(\underbrace{v-\sum_{i=1}^n a_iv_i, \underbrace{\sum_{i=1}^n a_i(v_i+U)}_{a_i=0 \quad \forall i \text{ (basis)}}}_{v=0})=(0,0)
    \end{align*}
    Therefore null$T=\{0\}$.
    
    \item \textbf{surjectivity}.
    Let for $u\in U$ we had $Tv = (u, \sum_{i=1}^n a_i(v_i+U)) \in U\times V/U$ where $v\in V$. It is clear that v is uniquely determined as $v=u+\sum_{i=1}^n a_iv_i$.
\end{itemize}
\vskip1mm\hrule

\subsubsection*{Exercise 17}
Let $v_1+U,\dots,v_n+U$ be a basis of $V/U$. Define the spanning list $W\triangleq span(v_1,\dots,v_n)$. Let $a_1,\dots,a_n\in\F$ such that
\begin{align*}
    a_1v_1+\dots+a_nv_n=0
\end{align*}
since $v_1+U,\dots,v_n+U$ are linearly independent, $a_1(v_1+U)+\dots+a_n(v_n+U)=a_1v_1+\dots+a_nv_n +U$ is zero only when $a_1=\dots=a_n=0$. Therefore $\dim W = \dim V/U$.\\

To have $V=U\oplus W$:
\begin{enumerate}
    \item $\mathbf{V= U+ W}$.

    For $v\in V$ exists $b_1,\dots,b_n\in\F$ such that $v+U=b_1(v_1+U)+\dots+b_n(v_n+U)$. Therefore $v-\sum_{i=1}^nb_iv_i\in U$. So
    \begin{align*}
        v = \underbrace{(v-\sum_{i=1}^nb_iv_i)}_{\in U} + \underbrace{\sum_{i=1}^nb_iv_i}_{\in W}
    \end{align*}
    meaning that $v\in U+W$ hence $V\subset U+W$. 

    Since $U$ and $W$ are subspaces of $V$, $U+W\subset V$.
    
    \item %$\mathbf{U\cap W=\{0\}}$.
    $\mathbf{\dim V = \dim U + \dim W}$.

    We showed $V=U+W$. We also know
    \begin{align*}
        \dim V/U &= \dim V -\dim U\\
        \dim W &=  \dim V -\dim U\\
        \dim V &=  \dim U +\dim W\\
    \end{align*}

    therefore $V=U\oplus W$.
    % Let $w\in U\cap W$, exists $a_1,\dots,a_n$ such that $w=a_1v_1+\dots+a_nv_n$, so $w+U=a_1(v_1+U)\dots+(a_nv_n+U)$.

    % On the other hand, since $w\in U$, then $w+U=0+U$, as a result $a_1=\dots=a_n=0$ which implies $w=0$.
\end{enumerate}
\vskip1mm\hrule

\end{document}

