\documentclass[12pt, letterpaper]{scrartcl}


\usepackage{fullpage} % Set margins and place page numbers at bottom center
\usepackage[shortlabels]{enumitem} % Use a. in the enumerate
\usepackage{amsmath} % aligned equations
\usepackage{graphicx} % include figure
\usepackage{float} % usage of H for figure float
\usepackage{amssymb} % \blacksqure
\usepackage{xhfill} % fill horizontal line

\usepackage{xcolor} % colors
\usepackage{sectsty} % section coloring
\sectionfont{\color{blue}}  % sets colour of sections

%%%%%%%%%%%%%%%%%%%%%%%%%%%%%%%%%%%%%%%%%%%%%%%%%%%%%%%%%%%%%%%%%%%%%%
% MY COMMANDS                                                        %
%%%%%%%%%%%%%%%%%%%%%%%%%%%%%%%%%%%%%%%%%%%%%%%%%%%%%%%%%%%%%%%%%%%%%%
\newcommand{\Z}{\mathbb{Z}}
\newcommand{\R}{\mathbb{R}}
\newcommand{\C}{\mathbb{C}}
\newcommand{\F}{\mathbb{F}}
\newcommand{\bigO}{\mathcal{O}}
\newcommand{\Real}{\mathcal{Re}}
\newcommand{\poly}{\mathcal{P}}
\newcommand{\mat}{\mathcal{M}}
\DeclareMathOperator{\Span}{span}
\newcommand{\Hom}{\mathcal{L}}
\DeclareMathOperator{\Null}{null}
\DeclareMathOperator{\Range}{range}
\newcommand{\defeq}{\vcentcolon=}
\newcommand{\restr}[1]{|_{#1}}


\begin{document}

% ### Header - start ###
\begin{center}
    \hrule
    \vspace{0.4cm}
    { \textbf{{\large Homework 5}} \\ MATH 543 --- Linear Algebra}
\end{center}
{ Name: \textbf{Ali Zafari} \hspace{\fill} Spring 2023 } \newline\hrule
% ### Header - end ###

\section*{5.B Eigenvectors and Upper-Triangular Matrices \xrfill[2pt]{3pt}[blue]}
\subsubsection*{Exercise 1}
\begin{enumerate}[(a)]
    \item For $I-T$ to be invertible, its composition (both right and left) with its inverse must be identity.
     \begin{align*}
         (I-T)(I+T+\dots+T^{n-1})v&=(I-T)(v+Tv+\dots+T^{n-1}v)\\
         &=v-Tv+Tv-T^2v+T^2v-\dots-T^{n-1}v+T^nv\\
         &=v
     \end{align*}
     and
      \begin{align*}
         (I+T+\dots+T^{n-1})(I-T)v&=(I+T+\dots+T^{n-1})(v-Tv)\\
         &=v-Tv+Tv-T^2v+T^2v-\dots+T^{n-1}v-T^nv\\
         &=v
     \end{align*}
     so $(I-T)$ is invertible with inverse $(I-T)^{-1}=(I+T+\dots+T^{n-1})$.
     \item
     Sum of a geometric sequence $1,r,r^2,\dots,r^{n-1}$ is:
     \begin{align*}
         1+r+r^2+\dots+r^{n-1}&=\frac{1-r^n}{1-r}\\
         1+r+r^2+\dots+r^{n-1}&=(1-r)^{-1} \qquad (r^n=0)
     \end{align*}
     replacing $r$ with $T$ and having $T^n=0$ resembles what we show in part (a).
\end{enumerate}
 
\vskip1mm\hrule


\subsubsection*{Exercise 3}
$T^2=I$ implies either $+1$ or $-1$ is an eigenvalue for $T$ (Exercise 5.A.22). Thus $+1$ is an eigenvalue for $T$ (hypothesis). Thus $\forall v \in V, v\neq 0$:
\begin{align*}
    Tv = v\\
    T=I
\end{align*}
\vskip1mm\hrule


\subsubsection*{Exercise 5}
$p\in\mathcal{P}(\F)$ can be written as
\begin{align*}
    p(z)&=a_0+a_1z+a_2z^2+\dots+a_mz^m\\
\end{align*}
where $a_0,\dots,a_m, z\in\F$, so
\begin{align*}
    p(T)&=a_0I+a_1T+a_2T^2+\dots+a_mT^m
\end{align*}
Then
\begin{align*}
    Sp(T)S^{-1}&=a_0SIS^{-1}+a_1STS^{-1}+a_2ST^2S^{-1}+\dots+a_mST^mS^{-1}\\
    &=a_0I+a_1STS^{-1}+a_2ST^2S^{-1}+\dots+a_mST^mS^{-1}\\
    &=a_0I+a_1STS^{-1}+a_2(STS^{-1})^2+\dots+a_m(STS^{-1})^m\\
    &=p(STS^{-1})
\end{align*}
where for the third equality we used $(STS^{-1})^m=ST^mS^{-1}$.
\vskip1mm\hrule


\subsubsection*{Exercise 13}
Consider 2 cases separately:\\
If $U\subset W$ is a zero subspace of $W$, there exists no eigenvalues for $T$ and the condition holds.\\
If $U\subset W$ is a non-zero finite-dimensional subspace, then $T_{|U}\in \mathcal{L}(U)$ has at least one eigenvalue (Theorem 5.21) which is a contradiction. Thus $U$ is infinite-dimensional.
\vskip1mm\hrule


\subsubsection*{Exercise 20}
By theorem 5.27, there exists basis $v_1, \dots, v_{\dim V}$ such that $T$ has an upper-triangular matrix with respect to it.

By theorem 5.26, $\Span(v_1, \dots, v_k)$ of dimension $k$ is invariant under $T$ for each $k\in\{1,\dots, {\dim V}\}$.
\vskip1mm\hrule

\clearpage
\section*{5.C Eigenspaces and Diagonal Matrices \xrfill[2pt]{3pt}[blue]}
\subsubsection*{Exercise 1}
Since $V$ is diagonalizable, there exists a diagonal matrix $\mathcal{M}(T)$ defined on a basis $v_1, \dots, v_n$. Thus $V$ is finite-dimensional.

Then for each $v_i$ exists a $\lambda_i$ such that $Tv_i=\lambda_iv_i$. Now we separate $\lambda_i$'s indices into two disjoint sets: $\lambda_i=0 \quad\forall i\in\{1,\dots, m\}$ and $\lambda_i\neq0 \quad\forall i\in\{m+1,\dots, n\}$.

 Thus 
\begin{align*}
    V=\Span(v_1,\dots,v_n)=\Span(v_1,\dots,v_m)\oplus \Span(v_{m+1},\dots,v_n)
\end{align*}
since $v_1, \dots, v_m, v_{m+1}, \dots, v_n$ is a basis. 
\begin{itemize}
    \item {\color{purple}$\Span(v_1,\dots,v_m)=\Null T$?}\\
    $\Span(v_1,\dots,v_m)=E(0,T)=\Null(T)$

    \item {\color{purple}$\Span(v_{m+1},\dots,v_n)=\Range T$?}\\
    For arbitrary $v\in V$ we have $v=a_1v_1+\dots+a_nv_n$ ($a_i\in \F$) then
    \begin{align*}
        Tv&=a_1Tv_1+\dots+a_nTv_n\\
        &=\underbrace{a_{1}Tv_{1}+\dots+a_mTv_m}_{=0}+\underbrace{a_{m+1}Tv_{m+1}+\dots+a_nTv_n}_{\in \Span(v_{m+1},\dots,v_n)}
    \end{align*}
    thus $\Range T\subset \Span(v_{m+1},\dots,v_n)$.
    
    On the other hand, for every $v_i$ where $i\in \{m+1, \dots, n\}$ we have $v_i=T(\frac{1}{\lambda_i}v_i)\in \Range T$ thus $\Span(v_{m+1},\dots,v_n) \subset\Range T$.
    
\end{itemize}
\vskip1mm\hrule


\subsubsection*{Exercise 2}
{\color{teal} Aside:

For any invertible $T\in\mathcal{L}(V)$, we have $V=\Null T\oplus\Range T$.

(\emph{proof.} invertible $T\in\mathcal{L}(V)$ implies $\Null T=\{0\}$ and $\Range T=V$ and obviously $\Null T\cap \Range T=\{0\}.$)
}\\
{\color{purple}\emph{Counterexample.}} Matrix of invertible $S\in\mathcal{L}(\R^2)$
\begin{align*}
    \mathcal{M}(S)= 
    \left[ 
    \begin{array}{cc}
        1 & 1\\ 0 & 1
    \end{array} 
    \right]
\end{align*}
is non-diagonal, although $\R^2=\Null S \oplus \Range S$.
\vskip1mm\hrule


\subsubsection*{Exercise 3}
\begin{enumerate}[(a)]
    \item $V=\Null T\oplus\Range T$
    \item $V=\Null T+\Range T$
    \item $\Null T\cap\Range T=\{0\}$
\end{enumerate}
\begin{enumerate}
    \item {\color{purple}$\mathbf{a\Rightarrow b}$} Trivial by definition of direct sum.

    \item {\color{purple}$\mathbf{b\Rightarrow c}$} Sum of two subspaces of finite-dimensional $V$ has dimension
    \begin{align*}
        \dim V&=\dim(\Null T)+\dim(\Range T)-\dim(\Null T\cap\Range T)\\
        &=\dim V-\dim(\Null T\cap\Range T)
    \end{align*} where Fundamental Theorem of Linear Maps is used for last equality. Thus $\dim(\Null T\cap\Range T)=0$ so $\Null T\cap\Range T=\{0\}$.

    \item {\color{purple}$\mathbf{c\Rightarrow a}$} For $\Null T+\Range T$ we have
    \begin{align*}
        \dim(\Null T+\Range T)&=\dim(\Null T)+\dim(\Range T)-\dim(\Null T\cap\Range T)\\
        &=\dim V-0
    \end{align*}
    thus $\Null T+\Range T=V$ and since $\Null T\cap\Range T=\{0\}$ the sum is direct.
\end{enumerate}
\vskip1mm\hrule


\subsubsection*{Exercise 5}
\begin{itemize}
    \item[$\Longrightarrow$]\mbox{}\\
    $T$ is diagonalizable. So there is a basis on which $\mathcal{M}(T)$ is diagonal.
    For $\lambda\in \C$:
    \begin{align*}
        \mathcal{M}(T-\lambda I)=\mathcal{M}(T)-\lambda \mathcal{M}(I)
    \end{align*}
    since both $\mathcal{M}(T)$ and $\mathcal{M}(I)$ are diagonal, then $\mathcal{M}(T-\lambda I)$ is diagonal.

    By using the result of Exercise 5.C.1 we have
    \begin{align*}
        V=\Null(T-\lambda I)\oplus\Range(T-\lambda I).
    \end{align*}
    
    \item[$\Longleftarrow$]\mbox{}\\
    ?
\end{itemize}
\vskip1mm\hrule


\subsubsection*{Exercise 8}
Suppose 2 and 6 are also eigenvalues of T,
\begin{align*}
    \dim E(8, T) + \dim E(6, T) + \dim E(2, T) &\leq \dim \F^5\\
    \dim E(6, T) + \dim E(2, T) &\leq 1
\end{align*}
then either $\dim E(6, T)=0$ or $\dim E(2, T)=0$. Thus either 6 or 2 is an NOT an eigenvalue of $T$, so either $T-6I$ or $T-2I$ is invertible, respectively, by theorem 5.6.
\vskip1mm\hrule


\subsubsection*{Exercise 10}
Let $0,\lambda_1,\dots,\lambda_m$ denote the full set of the distinct eigenvalues. Then 
\begin{align*}
    \underbrace{\dim E(0,T)}_{=\dim\Null T} + \dim E(\lambda_1, T) + \dots + \dim E(\lambda_m,T) &\leq \dim V\\
    \dim\Null T + \dim E(\lambda_1, T) + \dots + \dim E(\lambda_m,T) &\leq \dim\Null T + \dim\Range T\\
    \dim E(\lambda_1, T) + \dots + \dim E(\lambda_m,T) &\leq \dim\Range T
\end{align*}
where the RHS of 2nd inequality comes from Fundamental Theorem of Linear Maps.
\vskip1mm\hrule


\subsubsection*{Exercise 15}
Since $T$ is not diagonalizable, 8 cannot be an eigenvalue of it (Theorem 5.44). Thus $T-8I$ is invertible/injective/surjective (Theorem 5.6). By surjectivity for $v=(17, \sqrt{5}, 2\pi)$ there exists $(x,y,z)\in\C^3$ such that
\begin{align*}
    (T-8I)(x,y,z)&=(17, \sqrt{5}, 2\pi)\\
    T(x,y,z)&=(17+8x, \sqrt{5}+8y, 2\pi+8z)
\end{align*}
\vskip1mm\hrule


\subsubsection*{Exercise 16}
\begin{enumerate}[(a)]
    \item \emph{proof by induction.}\\
    $\mathbf{n=1}$: $T^1(0,1)=(1,1)=(F_1,F_2)$\\
    Assume it holds for $\mathbf{n\leq m-1}$.\\
    $\mathbf{n=m}$: $T^{m}(0,1)=TT^{m-1}(0,1)=T(F_{m-1},F_{m})=(F_{m},F_{m-1}+F_{m})=(F_{m},F_{m+1})$.

    \item 
    \begin{align*}
        T(x,y)&=(\lambda x,\lambda y)\\
        (y,x+y)&=(\lambda x,\lambda y)\\
        x+\lambda x &= \lambda (\lambda x)\\
        \lambda^2-\lambda-1&=0 \quad (x\neq0, y\neq0)\\
        \lambda&=\frac{1\pm\sqrt{5}}{2}
    \end{align*}

    \item 
    \begin{align*}
        E(\frac{1+\sqrt{5}}{2},T)&=\Null(T-(\frac{1+\sqrt{5}}{2}) I)\\
        &=\{(x,y)|(y-\frac{1+\sqrt{5}}{2}x, x+\frac{1-\sqrt{5}}{2}y )=0\}\\
        &=\{(x,y)|y=x(\frac{1+\sqrt{5}}{2})\}
    \end{align*}
    then we choose $v_1=(1, \frac{1+\sqrt{5}}{2})$.

    \begin{align*}
        E(\frac{1-\sqrt{5}}{2},T)&=\Null(T-(\frac{1-\sqrt{5}}{2}) I)\\
        &=\{(x,y)|(y-\frac{1-\sqrt{5}}{2}x, x+\frac{1+\sqrt{5}}{2}y )=0\}\\
        &=\{(x,y)|y=x(\frac{1-\sqrt{5}}{2})\}
    \end{align*}
    then we choose $v_2=(1, \frac{1-\sqrt{5}}{2})$.

    \item $(0,1)=\frac{1}{\sqrt{5}}v_1-\frac{1}{\sqrt{5}}v_2$ in terms of the new basis. Thus
    \begin{align*}
    T^n(0,1)&=(F_n,F_{n+1})\\
    &=T^n(\frac{1}{\sqrt{5}}v_1-\frac{1}{\sqrt{5}}v_2)\\
    &=\frac{1}{\sqrt{5}}(T^nv_1-T^nv_2)\\
    &=\frac{1}{\sqrt{5}}\left[\left(\frac{1+\sqrt{5}}{2}\right)^nv_1-\left(\frac{1-\sqrt{5}}{2}\right)^nv_2\right]\\
    &=\frac{1}{\sqrt{5}}\left[\left(\frac{1+\sqrt{5}}{2}\right)^n-\left(\frac{1-\sqrt{5}}{2}\right)^n, \left(\frac{1+\sqrt{5}}{2}\right)^{n+1}-\left(\frac{1-\sqrt{5}}{2}\right)^{n+1}\right]
    \end{align*}
    thus $F_n=\frac{1}{\sqrt{5}}\left[\left(\frac{1+\sqrt{5}}{2}\right)^n-\left(\frac{1-\sqrt{5}}{2}\right)^n\right]$.

    \item It suffices to show the magnitude of distance from $F_n$ to $\frac{1}{\sqrt{5}}\left(\frac{1+\sqrt{5}}{2}\right)^n$ is less than $\frac{1}{2}$. Let
    \begin{align*}
        d:&=\left|
        \frac{1}{\sqrt{5}}\left(\frac{1+\sqrt{5}}{2}\right)^n-\frac{1}{\sqrt{5}}\left(\frac{1-\sqrt{5}}{2}\right)^n - \frac{1}{\sqrt{5}}\left(\frac{1+\sqrt{5}}{2}\right)^n
        \right|\\
        &=
        \left|
        \frac{1}{\sqrt{5}}\left(\frac{1-\sqrt{5}}{2}\right)^n
        \right| \\
        &=
        \frac{1}{\sqrt{5}}
        \left|
        \frac{1-\sqrt{5}}{2}
        \right|^n\\
        &=
        \frac{1}{\sqrt{5}}
        \left|
        \frac{2}{1+\sqrt{5}}
        \right|^n
    \end{align*}
    be the distance.
    
    Since $\sqrt{5}>2$ then $\frac{1}{\sqrt{5}}<\frac{1}{2}$ and $\frac{2}{1+\sqrt{5}}<\frac{2}{3}$, thus
    \begin{align*}
        \frac{1}{\sqrt{5}}
        \left|
        \frac{2}{1+\sqrt{5}}
        \right|^n < \frac{1}{2}(\frac{2}{3})^n <\frac{1}{2}\frac{2}{3} <\frac{1}{2}
    \end{align*}
    
\end{enumerate}
\vskip1mm\hrule

\end{document}

